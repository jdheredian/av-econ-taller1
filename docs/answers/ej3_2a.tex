\begin{tcolorbox}
La variable de asignación al tratamiento se define como:
\[
D_i = \begin{cases} 1 & \text{si la mujer } i \text{ fue asignada al centro de salud con subsidio del 100\%} \\ 0 & \text{si la mujer } i \text{ fue asignada al centro de salud con subsidio del 10\%} \end{cases}
\]

donde $i$ refiere a cada mujer participante en el estudio. La variable toma dos valores posibles: $D_i \in \{0, 1\}$. La asignación fue determinada a nivel de centro de salud (no a nivel individual): todas las mujeres reclutadas en un mismo centro de salud recibieron el mismo nivel de subsidio. En la muestra de línea base, $N_1 = 4{,}313$ mujeres pertenecen al grupo tratado ($D_i = 1$) y $N_0 = 4{,}376$ al grupo de control ($D_i = 0$), para un total de $N = 8{,}689$ mujeres.
\end{tcolorbox}
