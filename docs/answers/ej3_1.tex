\begin{tcolorbox}
El estudio de Dupas et al. (2025) evaluó si el acceso gratuito a anticonceptivos reduce la fecundidad en Burkina Faso. El experimento incluyó 14,545 mujeres en 499 aldeas, distribuidas en 100 centros de salud y 20 provincias.
La aleatorización operó en dos niveles. Primero, los centros de salud se asignaron al azar a una de dos condiciones: subsidio completo, es decir 100\% del costo cubierto mediante vouchers o subsidio parcial, 10\% del costo cubierto, grupo de control. Esta asignación se realizó de forma estratificada por provincia, garantizando representación en las 20 provincias. Segundo, dentro de cada centro de salud, las mujeres elegibles fueron escogidas durante la encuesta de línea base (primavera 2018) e inscritas en el estudio. El seguimiento se realizó aproximadamente tres años después (primavera 2021).
El tratamiento principal fue el nivel de subsidio al costo de los anticonceptivos modernos es decir subsidio del 100\% (tratamiento) versus subsidio del 10\% (control). La elección del 10\% como umbral del grupo de control, en lugar de cero, respondió a la decisión ética de garantizar que todas las participantes tuvieran acceso mínimo a anticonceptivos.
Los autores identificaron tres desafíos principales. Primero, en julio de 2020, el gobierno de Burkina Faso anunció la gratuidad nacional de los anticonceptivos, lo que representa una posible violación del supuesto SUTVA. Segundo, la pérdida de seguimiento entre línea base y seguimiento fue del 13.8\%, con diferencia entre grupos (15.0\% en tratamiento vs. 12.6\% en control). Tercero, el diseño implicó múltiples variables de resultado, lo que exige cautela para evitar interpretaciones derivadas de comparaciones múltiples.
\end{tcolorbox}
