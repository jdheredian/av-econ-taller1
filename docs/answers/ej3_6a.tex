\begin{tcolorbox}

\begin{itemize}
    \item{Violación del supuesto SUTVA.} En julio de 2020, el gobierno de Burkina Faso anunció la gratuidad nacional de los anticonceptivos. Esto puede violar el supuesto SUTVA de dos formas. Primero, el ``valor único del tratamiento'' se ve comprometido ya que a partir del anuncio, las mujeres del grupo de control pueden acceder gratuitamente a anticonceptivos a través de canales gubernamentales, haciendo que $D_i = 0$ ya no represente exclusivamente el subsidio del 10\% del programa. Segundo, puede haber interferencia indirecta si la política pública afecta la oferta de anticonceptivos en toda la región, alterando la disponibilidad para todos los grupos.
    \item Si el grupo de control accede a anticonceptivos gratuitos tras el anuncio, la diferencia efectiva en acceso entre tratamiento y control se reduce. Esto atenúa el contraste entre grupos y sesga el estimador ITT hacia cero.
    \item El argumento basado en la implementación gradual de la política es parcialmente robusto. Dado que la política comenzó a implementarse de forma escalonada después del anuncio y el seguimiento terminó en primavera de 2021, el tiempo efectivo de exposición fue limitado. El rezago biológico de nueve meses para nacimientos refuerza que los nacimientos observados al cierre del seguimiento reflejan comportamientos anticonceptivos previos al anuncio. Sin embargo, si el uso de anticonceptivos en el grupo de control aumentó tras julio de 2020, el ITT para uso de anticonceptivos podría estar subestimado. El argumento es más sólido para el resultado de fecundidad que para el uso de anticonceptivos.
\end{itemize}

\end{tcolorbox}
