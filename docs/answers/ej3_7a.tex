\begin{tcolorbox}
La pregunta hace referencia al uso de vouchers como cuarta columna. Dado que en el inciso 3.5 se estimaron únicamente las tres primeras columnas de la Tabla 2 del artículo, el análisis se restringe a esos resultados: nacimientos durante el estudio, uso de anticoncepción moderna en los últimos tres años, y meses de uso moderno.

El uso de anticonceptivos modernos en los últimos tres años no aumentó ($\hat{\beta} = -0.006$, SE $= 0.017$, p $= 0.729$, media control $= 0.531$), ni los meses de uso moderno ($\hat{\beta} = -0.542$, SE $= 0.465$, p $= 0.244$, media control $= 9.414$). Ambos efectos son estadísticamente nulos.

El subsidio completo tampoco redujo los nacimientos ($\hat{\beta} = -0.013$, SE $= 0.013$, p $= 0.293$, media control $= 0.630$). Los tres nulos son coherentes entre sí: si el tratamiento no incrementó el uso de anticoncepción moderna, no hay canal mediante el cual pudiera reducir la fecundidad. Una explicación es la posible sustitución: las mujeres asignadas al subsidio completo canjearon vouchers pero eran, en su mayoría, mujeres que ya usaban anticonceptivos modernos y que simplemente sustituyeron su canal de obtención habitual. El subsidio del 10\% en el grupo de control fue suficiente para que las mujeres motivadas ya accedieran, de modo que el subsidio adicional no incorporó nuevas usuarias ni redujo nacimientos.
\end{tcolorbox}
