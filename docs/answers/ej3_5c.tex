\begin{tcolorbox}
Los efectos ITT estimados para los tres resultados principales (columnas 1--3 de la Tabla 2 del artículo, excluyendo la columna 4 de uso de voucher) se presentan a continuación:

\begin{center}
\small
\setlength{\tabcolsep}{6pt}
\begin{tabular}{lccc}
\hline\hline
 & (1) & (2) & (3) \\
 & Nacimiento & Usó anticoncepción & Meses de uso \\
 & durante el estudio & moderna en 3 años & moderno (últ.\ episodio) \\
\hline
\multicolumn{4}{l}{\textit{Panel A: solo controles de línea final}} \\[2pt]
Subsidio 100\% & $-0.013$ & $-0.006$ & $-0.542$ \\
               & $(0.013)$ & $(0.017)$ & $(0.465)$ \\
Controles línea base & No & No & No \\
Efectos fijos provincia & Sí & Sí & Sí \\
Observaciones & $7{,}489$ & $7{,}241$ & $7{,}489$ \\
Media control (10\%) & 0.630 & 0.531 & 9.414 \\
\hline\hline
\multicolumn{4}{l}{\footnotesize *** p$<$0.01, ** p$<$0.05, * p$<$0.1.} \\
\multicolumn{4}{l}{\footnotesize SE clusterizados a nivel de CSPS entre paréntesis.} \\
\hline
\end{tabular}
\end{center}

Los tres resultados corresponden a las columnas 1 a 3 de la Tabla 2 del artículo: (1) si la mujer tuvo al menos un nacimiento durante el período de estudio, (2) si usó algún anticonceptivo moderno en los últimos tres años, y (3) los meses de uso moderno del último episodio por método (cero para quienes no usaron). Para cada resultado se estimó la ecuación con efectos fijos de provincia y errores estándar clusterizados a nivel de CSPS, sin controles adicionales de línea base (equivalente al Panel A del artículo).

Ninguno de los tres efectos es estadísticamente significativo: nacimientos $\hat{\beta} = -0.013$ (SE $= 0.013$, p $= 0.293$), uso moderno en 3 años $\hat{\beta} = -0.006$ (SE $= 0.017$, p $= 0.729$), meses de uso $\hat{\beta} = -0.542$ (SE $= 0.465$, p $= 0.244$). Los resultados son cualitativamente consistentes con la Tabla 2 del artículo; las diferencias en magnitud obedecen al uso de la submuestra del 60\%.
\end{tcolorbox}
