\begin{tcolorbox}
\begin{enumerate}
    \item Efecto Hawthorne: Las mujeres cambian su comportamiento por el hecho de saber que están siendo observadas, independientemente del tratamiento recibido. En este contexto, mujeres de ambos grupos podrían declarar un mayor uso de anticonceptivos o adoptar prácticas de planificación familiar al saber que participan en un estudio de salud reproductiva. Esto sesgaría el coeficiente hacia cero, dado que eleva el comportamiento en el grupo de control y reduce la diferencia observada entre grupos.

    \item Efecto John Henry: El grupo de control, consciente de que recibe menos subsidio, compensa adoptando con mayor esfuerzo métodos alternativos (anticonceptivos tradicionales, abstinencia) para no quedar en desventaja. Lo anterior, reduciría la diferencia en uso de anticonceptivos y fecundidad entre grupos, sesgando el ITT hacia cero.

    \item Efecto de demanda del experimentador: Las mujeres intuyen que los investigadores esperan que el subsidio reduzca la fecundidad y responden de acuerdo con esa expectativa. En el grupo tratado, podrían sobrereportar uso de anticonceptivos; en el control, no. Por esto, se estaría sesgando el coeficiente estimado hacia abajo (más negativo) para uso de anticonceptivos, sobreestimando el efecto real.

    \item Efecto placebo: El hecho de ser incluidas en el programa, aun en el grupo de control con subsidio del 10\%, genera cambios en comportamiento (mayor concientización sobre planificación familiar). Esto eleva el uso de anticonceptivos en el grupo de control, reduciendo la diferencia con el tratamiento y sesgando el ITT hacia cero.
\end{enumerate}
\end{tcolorbox}
