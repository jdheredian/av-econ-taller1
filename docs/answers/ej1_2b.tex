\begin{tcolorbox}
Partiendo del mismo punto de la demostración anterior:
\begin{align*}
\tau_{\text{ing}}
  &= \mathbb{E}[Y_i(1) \mid D_i = 1] - \mathbb{E}[Y_i(0) \mid D_i = 0].
\end{align*}

Se suma y resta un contrafactual, concretamente $\mathbb{E}[Y_i(1) \mid D_i = 0]$:
\begin{align*}
\tau_{\text{ing}}
  &= \mathbb{E}[Y_i(1) \mid D_i = 1] - \mathbb{E}[Y_i(1) \mid D_i = 0]
     + \mathbb{E}[Y_i(1) \mid D_i = 0] - \mathbb{E}[Y_i(0) \mid D_i = 0].
\end{align*}

Reeorganizando, se ve que el segundo par de términos es el ATU por definición:
\[
\text{ATU} = \mathbb{E}[Y_i(1) - Y_i(0) \mid D_i = 0]
           = \mathbb{E}[Y_i(1) \mid D_i = 0] - \mathbb{E}[Y_i(0) \mid D_i = 0].
\]

Y el primer par de términos es el sesgo de selección en $Y_i(1)$:
\[
\text{Sesgo en } Y_i(1) = \mathbb{E}[Y_i(1) \mid D_i = 1] - \mathbb{E}[Y_i(1) \mid D_i = 0].
\]

Reordenando:
\[
\tau_{\text{ing}} = \underbrace{\mathbb{E}[Y_i(1) \mid D_i = 0] - \mathbb{E}[Y_i(0) \mid D_i = 0]}_{\text{ATU}} + \underbrace{\mathbb{E}[Y_i(1) \mid D_i = 1] - \mathbb{E}[Y_i(1) \mid D_i = 0]}_{\text{Sesgo de selección en } Y_i(1)}.
\]

Este sesgo captura si los tratados obtendrían puntajes con el programa distintos a los del grupo de control. Si los estudiantes del aula tecnológica, inherentemente, tuvieran puntajes con tratamiento más altos por capacidad propia, $\tau_{\text{ing}}$ sobreestima el ATU.
\end{tcolorbox}
