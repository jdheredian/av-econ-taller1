\begin{tcolorbox}
La base de endline se emparejó con la línea base usando el identificador de mujer. Las observaciones presentes en la línea base pero ausentes en el seguimiento se clasifican como atrición. Los resultados son:

\begin{center}
\begin{tabular}{lrr}
\hline
Grupo & Atrición (N) & Tasa de atrición (\%) \\
\hline
Tratamiento (100\%) & 649 & 15.04\% \\
Control (10\%) & 551 & 12.59\% \\
\textbf{Total} & \textbf{1{,}200} & \textbf{13.81\%} \\
\hline
\end{tabular}
\end{center}

La muestra de línea base contiene $N = 8{,}689$ mujeres (4{,}313 en tratamiento, 4{,}376 en control). Tras el merge, la muestra de seguimiento disponible es de aproximadamente $7{,}489$ observaciones para el resultado principal (nacimientos). La tasa de atrición diferencial entre grupos (15.0\% en tratamiento vs. 12.6\% en control) representa una amenaza potencial a la validez interna del estudio: si las mujeres que abandonaron el seguimiento son sistemáticamente distintas entre grupos, las estimaciones ITT podrían estar sesgadas. Los autores analizan este problema y reportan que los resultados son robustos a distintos supuestos sobre las mujeres que dejaron de reportarse.
\end{tcolorbox}
