\begin{tcolorbox}[breakable]
Los resultados nulos estimados son consistentes y robustos. Los tres efectos ITT apuntan en la misma dirección: nacimientos ($\hat{\beta} = -0.013$, p $= 0.293$), uso moderno en tres años ($\hat{\beta} = -0.006$, p $= 0.729$) y meses de uso moderno ($\hat{\beta} = -0.542$, p $= 0.244$), y el intervalo de confianza al 95\% sobre nacimientos $[-0.038,\; 0.011]$ descarta reducciones superiores a 3.8 pp (6.0\% de la media del grupo de control), coherente con el cálculo ex post del artículo que establece una potencia mínima detectable de 3.4 pp. Dupas et al.\ (2025) extienden estos nulos con verificaciones adicionales: el efecto sobre embarazos es $-1.9$ pp (no significativo), lo que descarta el canal de abortos espontáneos. La evidencia sobre el mecanismo refuerza los nulos: solo el 17\% del grupo de tratamiento canjeó el voucher frente al 14\% del control (diferencia de 3 pp, significativa al 1\%), y la mayoría ya usaba anticoncepción moderna antes del estudio, por lo que el subsidio sustituyó su canal de compra sin incorporar nuevas usuarias. Los nulos persisten en todos los subgrupos con mayor demanda esperada: mujeres con necesidad insatisfecha, mujeres que declaran no poder pagar (mayor efecto: $-2.5$ pp, aun no significativo), y mujeres cercanas al centro de salud ($-3.2$ pp, no significativo tras corrección por hipótesis múltiples). Finalmente, el artículo combina el subsidio con debates comunitarios sobre normas sociales y corrección de percepciones sobre mortalidad infantil, y ninguna interacción es significativa al 5\%, lo que confirma que la barrera no es de precio ni de información, sino de preferencias: los hombres desean en promedio 9 hijos y las mujeres 6. 
\end{tcolorbox}
