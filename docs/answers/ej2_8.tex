\begin{tcolorbox}
La recomendación del árbitro \#2 no es correcta. La simulación muestra que clusterizar a nivel de colegio, cuando la fuente de variación es heterogeneidad en el efecto del tratamiento entre colegios y no correlación intra-cluster en los errores, produce sobrecobertura del 100\% frente al 94.7\% del estimador robusto a heterocedasticidad. La varianza estimada con cluster es 440 veces mayor que la robusta. En términos prácticos, aplicar la regla del árbitro llevaría a reportar errores estándar inflados, reducir el poder estadístico para detectar efectos verdaderos y, en experimentos como el de Gershenson et al. (2022) donde el tratamiento se aleatorizó dentro de colegios, reportar inferencia excesivamente conservadora.

La decisión de clusterizar debe estar motivada por el diseño del estudio y la presencia efectiva de correlación intra-cluster en los errores, no por una regla de maximizar el nivel de agregación. Abadie et al. (2022) establecen que la clusterización es apropiada cuando el proceso que genera los datos produce correlación entre las observaciones dentro de un grupo. En este experimento, dado que la aleatorización fue dentro de cada colegio, clusterizar a nivel de colegio es la elección justificada por el diseño; clusterizar a niveles más agregados, sin evidencia de correlación a esos niveles, solo infla los errores sin ganar validez de la inferencia.
\end{tcolorbox}
