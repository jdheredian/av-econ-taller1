\begin{tcolorbox}
La decisión de usar un subsidio del 10\% como grupo de control, en lugar de no ofrecer ningún subsidio, responde a una razón ética: dado que los anticonceptivos modernos tienen beneficios de salud documentados, privar completamente de acceso a las participantes del grupo de control habría sido éticamente cuestionable. El subsidio del 10\% garantiza un nivel mínimo de acceso para todas las mujeres.

Esta decisión tiene consecuencias directas sobre el parámetro causal que se estima. El parámetro identificado es el {efecto ITT del subsidio completo relativo al subsidio parcial} $\tau = \mathbb{E}[Y_i(1) - Y_i(0)]$, donde $Y_i(1)$ y $Y_i(0)$ corresponden al resultado bajo subsidio del 100\% y del 10\%, respectivamente. El experimento no identifica el efecto del acceso a anticonceptivos comparado con ningún acceso (el ATE absoluto frente a un grupo sin ningún subsidio).
\end{tcolorbox}
