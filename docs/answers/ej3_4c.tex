\begin{tcolorbox}
Los resultados de la prueba de balance no muestran evidencia de desbalance sistemático entre los grupos de tratamiento y control. El estadístico $F$ conjunto es $F = 1.705$ con un p-valor de $0.117$, por lo que no se rechaza la hipótesis nula de que todos los coeficientes son simultáneamente iguales a cero al nivel convencional del 5\%.

A nivel individual, solo la variable ``No podría pagar por anticonceptivos'' presenta un coeficiente marginalmente significativo ($\hat{\beta} = 0.032$, p-valor $= 0.065$) al 10\%, lo que indica que el grupo tratado reporta una probabilidad ligeramente mayor de no poder pagar en la línea de base. Dado que se trata de una variable de línea de base y que la diferencia se observa en apenas una de las siete variables con significancia marginal, puede atribuirse a variación muestral aleatoria.

Estos resultados son coherentes con la estratificación por provincia y con el proceso de aleatorización a nivel de centro de salud. La comparación con la Tabla A.1 del artículo original muestra patrones similares (ninguna diferencia estadísticamente significativa al 5\%), lo que confirma que la submuestra del 60\% preserva el balance del diseño experimental. Se concluye que la aleatorización fue implementada correctamente.
\end{tcolorbox}
