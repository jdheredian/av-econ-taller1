\begin{tcolorbox}
\textbf{Homocedasticidad:} El supuesto dice que $\mathbb{V}[\varepsilon_i \mid X_i] = \sigma^2$ para todo $i$; es decir, la varianza del error es constante e independiente del valor de las caracteristicas observables o variables independientes. En este contexto, $X_i = D_i$ indica si el estudiante $i$ tuvo un profesor negro en kínder y los errores recogen todos las caracteristicas no observables que afectan el resultado ($Y_i$). El supuesto implica que estas caracteristicas no observables que tienen relación en el puntaje SAT o en la asistencia universitaria tienen la misma dispersión para los estudiantes con y sin profesor negro. Esto es debatible. Los estudiantes cuyo profesor en kínder fue negro pueden tener caracteristicas no observables qué estén afectando los resultados de las variables de estudio como entornos distintos, como colegios con mayor movilidad social, lo que hace que la variabilidad de los resultados difiera frente a la del grupo de control.

\textbf{No correlación:} El supuesto establece que $\mathbb{E}[\varepsilon_i \varepsilon_j \mid X_i, X_j] = 0$ para todo $i \neq j$; es decir, los errores de distintos individuos no están relacionados entre sí. Así, los estudiantes del mismo colegio comparten recursos, docentes y entorno socioeconómico. Un colegio con factores no observados favorables, como financiación adicional, se esperaría que aumentara el resultado de todos sus estudiantes de manera simultánea. Esto produce correlación positiva entre los errores dentro del mismo colegio, lo que viola el supuesto de no correlación. 
\end{tcolorbox}
