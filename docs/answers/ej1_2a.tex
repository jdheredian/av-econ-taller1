\begin{tcolorbox}
Partiendo de la definición del estimador de diferencia de medias:
\begin{align*}
\tau_{\text{ing}}
  &= \mathbb{E}[Y_i \mid D_i = 1] - \mathbb{E}[Y_i \mid D_i = 0].
\end{align*}
Se reemplaza la ecuación  $Y_i = D_i \times Y_i(1) + (1-D_i) \times Y_i(0)$ dentro del estimador:
\begin{align*}
  \tau_{\text{ing}}
  &= \mathbb{E}[D_i \times Y_i(1) + (1-D_i) \times Y_i(0) \mid D_i = 1] 
  \\&- \mathbb{E}[D_i \times Y_i(1) + (1-D_i) \times Y_i(0) \mid D_i = 0]
  \\
    \tau_{\text{ing}}
  &= \mathbb{E}[(1) \times Y_i(1) + (1-(1)) \times Y_i(0) \mid D_i = 1] 
  \\&- \mathbb{E}[(0) \times Y_i(1) + (1-(0)) \times Y_i(0) \mid D_i = 0]
  \\
\tau_{\text{ing}}
  &= \mathbb{E}[Y_i(1) \mid D_i = 1] - \mathbb{E}[Y_i(0) \mid D_i = 0]
\end{align*}

Se suma y resta un contrafactual, concretamente $\mathbb{E}[Y_i(0) \mid D_i = 1]$ :
\begin{align*}
\tau_{\text{ing}}
  &= \mathbb{E}[Y_i(1) \mid D_i = 1] - \mathbb{E}[Y_i(0) \mid D_i = 1]
     + \mathbb{E}[Y_i(0) \mid D_i = 1] - \mathbb{E}[Y_i(0) \mid D_i = 0].
\end{align*}

Reordenando, el primer par de términos es el ATT por definición:
\[
\text{ATT} = \mathbb{E}[Y_i(1) - Y_i(0) \mid D_i = 1]
           = \mathbb{E}[Y_i(1) \mid D_i = 1] - \mathbb{E}[Y_i(0) \mid D_i = 1].
\]

El segundo par de términos es el sesgo de selección:
\[
\text{Sesgo} = \mathbb{E}[Y_i(0) \mid D_i = 1] - \mathbb{E}[Y_i(0) \mid D_i = 0].
\]

Por tanto:
\[
\tau_{\text{ing}} = \underbrace{\mathbb{E}[Y_i(1) \mid D_i = 1] - \mathbb{E}[Y_i(0) \mid D_i = 1]}_{\text{ATT}} + \underbrace{\mathbb{E}[Y_i(0) \mid D_i = 1] - \mathbb{E}[Y_i(0) \mid D_i = 0]}_{\text{Sesgo de selección}}.
\]

El sesgo de selección captura la diferencia en el resultado potencial sin tratamiento entre los que efectivamente se tratan y los que no. Si los estudiantes asignados al aula tecnológica hubieran tenido, en ausencia del programa, puntajes distintos a los del salón regular, entonces $\tau_{\text{ing}} \neq \text{ATT}$.
\end{tcolorbox}
